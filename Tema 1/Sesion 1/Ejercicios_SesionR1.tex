% Options for packages loaded elsewhere
\PassOptionsToPackage{unicode}{hyperref}
\PassOptionsToPackage{hyphens}{url}
%
\documentclass[
]{article}
\usepackage{lmodern}
\usepackage{amssymb,amsmath}
\usepackage{ifxetex,ifluatex}
\ifnum 0\ifxetex 1\fi\ifluatex 1\fi=0 % if pdftex
  \usepackage[T1]{fontenc}
  \usepackage[utf8]{inputenc}
  \usepackage{textcomp} % provide euro and other symbols
\else % if luatex or xetex
  \usepackage{unicode-math}
  \defaultfontfeatures{Scale=MatchLowercase}
  \defaultfontfeatures[\rmfamily]{Ligatures=TeX,Scale=1}
\fi
% Use upquote if available, for straight quotes in verbatim environments
\IfFileExists{upquote.sty}{\usepackage{upquote}}{}
\IfFileExists{microtype.sty}{% use microtype if available
  \usepackage[]{microtype}
  \UseMicrotypeSet[protrusion]{basicmath} % disable protrusion for tt fonts
}{}
\makeatletter
\@ifundefined{KOMAClassName}{% if non-KOMA class
  \IfFileExists{parskip.sty}{%
    \usepackage{parskip}
  }{% else
    \setlength{\parindent}{0pt}
    \setlength{\parskip}{6pt plus 2pt minus 1pt}}
}{% if KOMA class
  \KOMAoptions{parskip=half}}
\makeatother
\usepackage{xcolor}
\IfFileExists{xurl.sty}{\usepackage{xurl}}{} % add URL line breaks if available
\IfFileExists{bookmark.sty}{\usepackage{bookmark}}{\usepackage{hyperref}}
\hypersetup{
  pdftitle={Introducción al lenguaje R},
  pdfauthor={Grado Ciencia de Datos},
  hidelinks,
  pdfcreator={LaTeX via pandoc}}
\urlstyle{same} % disable monospaced font for URLs
\usepackage[margin=1in]{geometry}
\usepackage{color}
\usepackage{fancyvrb}
\newcommand{\VerbBar}{|}
\newcommand{\VERB}{\Verb[commandchars=\\\{\}]}
\DefineVerbatimEnvironment{Highlighting}{Verbatim}{commandchars=\\\{\}}
% Add ',fontsize=\small' for more characters per line
\usepackage{framed}
\definecolor{shadecolor}{RGB}{248,248,248}
\newenvironment{Shaded}{\begin{snugshade}}{\end{snugshade}}
\newcommand{\AlertTok}[1]{\textcolor[rgb]{0.94,0.16,0.16}{#1}}
\newcommand{\AnnotationTok}[1]{\textcolor[rgb]{0.56,0.35,0.01}{\textbf{\textit{#1}}}}
\newcommand{\AttributeTok}[1]{\textcolor[rgb]{0.77,0.63,0.00}{#1}}
\newcommand{\BaseNTok}[1]{\textcolor[rgb]{0.00,0.00,0.81}{#1}}
\newcommand{\BuiltInTok}[1]{#1}
\newcommand{\CharTok}[1]{\textcolor[rgb]{0.31,0.60,0.02}{#1}}
\newcommand{\CommentTok}[1]{\textcolor[rgb]{0.56,0.35,0.01}{\textit{#1}}}
\newcommand{\CommentVarTok}[1]{\textcolor[rgb]{0.56,0.35,0.01}{\textbf{\textit{#1}}}}
\newcommand{\ConstantTok}[1]{\textcolor[rgb]{0.00,0.00,0.00}{#1}}
\newcommand{\ControlFlowTok}[1]{\textcolor[rgb]{0.13,0.29,0.53}{\textbf{#1}}}
\newcommand{\DataTypeTok}[1]{\textcolor[rgb]{0.13,0.29,0.53}{#1}}
\newcommand{\DecValTok}[1]{\textcolor[rgb]{0.00,0.00,0.81}{#1}}
\newcommand{\DocumentationTok}[1]{\textcolor[rgb]{0.56,0.35,0.01}{\textbf{\textit{#1}}}}
\newcommand{\ErrorTok}[1]{\textcolor[rgb]{0.64,0.00,0.00}{\textbf{#1}}}
\newcommand{\ExtensionTok}[1]{#1}
\newcommand{\FloatTok}[1]{\textcolor[rgb]{0.00,0.00,0.81}{#1}}
\newcommand{\FunctionTok}[1]{\textcolor[rgb]{0.00,0.00,0.00}{#1}}
\newcommand{\ImportTok}[1]{#1}
\newcommand{\InformationTok}[1]{\textcolor[rgb]{0.56,0.35,0.01}{\textbf{\textit{#1}}}}
\newcommand{\KeywordTok}[1]{\textcolor[rgb]{0.13,0.29,0.53}{\textbf{#1}}}
\newcommand{\NormalTok}[1]{#1}
\newcommand{\OperatorTok}[1]{\textcolor[rgb]{0.81,0.36,0.00}{\textbf{#1}}}
\newcommand{\OtherTok}[1]{\textcolor[rgb]{0.56,0.35,0.01}{#1}}
\newcommand{\PreprocessorTok}[1]{\textcolor[rgb]{0.56,0.35,0.01}{\textit{#1}}}
\newcommand{\RegionMarkerTok}[1]{#1}
\newcommand{\SpecialCharTok}[1]{\textcolor[rgb]{0.00,0.00,0.00}{#1}}
\newcommand{\SpecialStringTok}[1]{\textcolor[rgb]{0.31,0.60,0.02}{#1}}
\newcommand{\StringTok}[1]{\textcolor[rgb]{0.31,0.60,0.02}{#1}}
\newcommand{\VariableTok}[1]{\textcolor[rgb]{0.00,0.00,0.00}{#1}}
\newcommand{\VerbatimStringTok}[1]{\textcolor[rgb]{0.31,0.60,0.02}{#1}}
\newcommand{\WarningTok}[1]{\textcolor[rgb]{0.56,0.35,0.01}{\textbf{\textit{#1}}}}
\usepackage{graphicx,grffile}
\makeatletter
\def\maxwidth{\ifdim\Gin@nat@width>\linewidth\linewidth\else\Gin@nat@width\fi}
\def\maxheight{\ifdim\Gin@nat@height>\textheight\textheight\else\Gin@nat@height\fi}
\makeatother
% Scale images if necessary, so that they will not overflow the page
% margins by default, and it is still possible to overwrite the defaults
% using explicit options in \includegraphics[width, height, ...]{}
\setkeys{Gin}{width=\maxwidth,height=\maxheight,keepaspectratio}
% Set default figure placement to htbp
\makeatletter
\def\fps@figure{htbp}
\makeatother
\setlength{\emergencystretch}{3em} % prevent overfull lines
\providecommand{\tightlist}{%
  \setlength{\itemsep}{0pt}\setlength{\parskip}{0pt}}
\setcounter{secnumdepth}{-\maxdimen} % remove section numbering

\title{Introducción al lenguaje R}
\usepackage{etoolbox}
\makeatletter
\providecommand{\subtitle}[1]{% add subtitle to \maketitle
  \apptocmd{\@title}{\par {\large #1 \par}}{}{}
}
\makeatother
\subtitle{Primera Sesion.}
\author{Grado Ciencia de Datos}
\date{}

\begin{document}
\maketitle

\hypertarget{generaciuxf3n-de-guiuxf3n-de-ejercicios}{%
\section{Generación de Guión de
Ejercicios}\label{generaciuxf3n-de-guiuxf3n-de-ejercicios}}

Compila este documento en formato HTML y genera el documento de los
ejercicios (todavía sin completar) de la primera sesión de Introducción
al Lenguaje R.

\hypertarget{r-como-calculadora}{%
\section{R Como Calculadora}\label{r-como-calculadora}}

\hypertarget{determina-el-valor-resultante-de-las-siguientes-operaciones}{%
\subsection{Determina el valor resultante de las siguientes
operaciones}\label{determina-el-valor-resultante-de-las-siguientes-operaciones}}

Puedes consultar la ayuda de las funciones aritméticas
\texttt{?Arithmetic} y trigonométicas \texttt{?Trig}

\begin{itemize}
\tightlist
\item
  (1+(2-3)*5)/4
\item
  La exponencial de 0
\item
  Calcula el coseno del doble de \(\pi\)
\item
  El seno de un cuarto de \(\pi\)
\item
  El valor de la raiz cuadrada de 2 partido por 2
\item
  Resta los dos últimos valores calculados ¿qué observas?
\end{itemize}

\hypertarget{interpreta-el-resultado-de-las-siguientes-operaciones}{%
\subsection{Interpreta el resultado de las siguientes
operaciones}\label{interpreta-el-resultado-de-las-siguientes-operaciones}}

\begin{itemize}
\tightlist
\item
  log(0)
\item
  0/0
\item
  cos(0)/sin(0)
\end{itemize}

\hypertarget{determina-el-valor-resultante-de-las-siguientes-expresiones}{%
\subsection{Determina el valor resultante de las siguientes
expresiones}\label{determina-el-valor-resultante-de-las-siguientes-expresiones}}

Necesitarás compilar el documento para visualizar las expresiones
escritas en lenguaje
\({\displaystyle \mathbf {L\!\!^{{}_{\scriptstyle A}}\!\!\!\!\!\;\;T\!_{\displaystyle E}\!X} }\)

\begin{itemize}
\tightlist
\item
  \(2^{2*8}\)
\item
  \(e^{-\infty}\)
\item
  \(\frac{3}{2}\)
\item
  \(\frac{1}{1+\infty}\)
\item
  \(\left({1+\displaystyle\frac{1}{10^5}}\right)^{10^5}\)
\end{itemize}

\hypertarget{realiza-las-siguientes-operaciones-luxf3gicas}{%
\subsection{Realiza las siguientes operaciones
lógicas}\label{realiza-las-siguientes-operaciones-luxf3gicas}}

Puedes consultar la ayuda de los operadores relacionales
\texttt{?Comparison}

\begin{itemize}
\tightlist
\item
  Es 3 menor o igual que 7
\item
  El \(cos(\pi)\) es diferente del seno de \(sen(\pi)\)
\item
  El \(cos(\pi)\) es mayor, igual o menor que seno de \(sen(\pi)\)
\end{itemize}

\hypertarget{operaciones-muxfaltiples}{%
\subsection{Operaciones Múltiples}\label{operaciones-muxfaltiples}}

Sabiendo que se pueden generar secuencias numéricas (después los
llamaremos vectores) y operar con ellas para agilizar los cálculos.
Interpreta el resultado de las siguiente operaciones de secuencias.

\begin{itemize}
\tightlist
\item
  1:5+1
\item
  1:5+1:5
\item
  1:(5*5)
\item
  1:5*5
\item
  sum(1:5)
\item
  prod(1:5)
\item
  sum(1:5+1)
\end{itemize}

Interpreta el resultado de la siguiente instrucción

\begin{Shaded}
\begin{Highlighting}[]
\KeywordTok{log}\NormalTok{(}\OperatorTok{-}\DecValTok{1}\OperatorTok{:}\DecValTok{3}\NormalTok{) }
\end{Highlighting}
\end{Shaded}

\begin{verbatim}
## Warning in log(-1:3): Se han producido NaNs
\end{verbatim}

\begin{verbatim}
## [1]       NaN      -Inf 0.0000000 0.6931472 1.0986123
\end{verbatim}

\hypertarget{variables}{%
\section{Variables}\label{variables}}

Ejecuta las siguientes líneas de código paso a paso (CTRL+Enter).
Observa como las siguientes líneas de código generan las variables A, B
y C y su valor aparece en el \emph{Workspace Global Environment}.

\begin{Shaded}
\begin{Highlighting}[]
\NormalTok{A<-}\DecValTok{3}
\NormalTok{B<-A}\OperatorTok{+}\DecValTok{7}
\NormalTok{C<-A}\OperatorTok{*}\NormalTok{B}
\end{Highlighting}
\end{Shaded}

Utiliza la instrucción \texttt{ls} y determina la función que realiza.

Busca información de la instrucción \texttt{rm} y elimina la variable A.
¿qué pasa si intentas eliminar dos veces la variable A?

Combina las intrucciones \texttt{ls} y \texttt{rm} para eliminar todas
las variables (y objetos) del \emph{workspace}. Observa que la
instrucción \texttt{rm} tiene un argumento denominado \texttt{list} para
introduccir las variables a eliminar.

¿Por qué el siguiente código da error pero obtenemos el valor de la
variable A después de eliminarla? Comprueba lo que hacen estas líneas
sobre el \texttt{workspace}. Ejecuta las líneas de una en una.

\begin{Shaded}
\begin{Highlighting}[]
\NormalTok{A<-}\DecValTok{3}
\KeywordTok{rm}\NormalTok{(a)}
\end{Highlighting}
\end{Shaded}

\begin{verbatim}
## Warning in rm(a): objeto 'a' no encontrado
\end{verbatim}

\begin{Shaded}
\begin{Highlighting}[]
\KeywordTok{print}\NormalTok{(A)}
\end{Highlighting}
\end{Shaded}

\begin{verbatim}
## [1] 3
\end{verbatim}

\hypertarget{rmarkdown---imuxe1genes}{%
\section{RMarkdown - imágenes}\label{rmarkdown---imuxe1genes}}

\hypertarget{introduce-entre-las-luxedneas-horizontales-siguientes-las-marcas-una-imagen-que-te-guste-y-su-descripciuxf3n.}{%
\subsection{Introduce entre las líneas horizontales siguientes (las
marcas ****) una imagen que te guste y su
descripción.}\label{introduce-entre-las-luxedneas-horizontales-siguientes-las-marcas-una-imagen-que-te-guste-y-su-descripciuxf3n.}}

Pasos: * Obtén una imagen de Internet que desees comentar. * Coloca la
imagen en el directorio de trabajo para que sea accesible fácilmente
accesible. * Crea un título para la imágen de nivel 3, es decir con
\#\#\# * Enlaza la imagen para que aparezca en el documento. * Introduce
la descripción de la imágen utilizando \texttt{Blockquotes} de Markdown
* Añade un enlace a la página de dónde has sacado la imagen.

\begin{center}\rule{0.5\linewidth}{0.5pt}\end{center}

\begin{center}\rule{0.5\linewidth}{0.5pt}\end{center}

\hypertarget{generaciuxf3n-de-guiuxf3n-de-ejercicios-completos}{%
\section{Generación de Guión de Ejercicios
Completos}\label{generaciuxf3n-de-guiuxf3n-de-ejercicios-completos}}

Compila este documento en formato HTML y genera el documento de los
ejercicios (¡ahora completados!) de la primera sesión de Introducción al
Lenguaje R.

\end{document}
